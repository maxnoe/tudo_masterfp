\section{Messvorgang}

Es lagen eine Reihe von unterschiedlichen Koaxialkabeln vor. Bis auf eines hatten alle eine Eingangsimpedanz von \SI{50}{\ohm}.
Ein LCR Messgerät zur Vermessung der Kabeleigenschaften wird direkt an das zu untersuchende Kabel angeschlossen.
Über einen Frequenzgenerator können signale in die Kabel eingespeist werden.
Dieser Arbeitet in einem Frequenzbereich von \SIrange{100}{100000}{\hertz}. Dieser wurde für die Messung von L, C, und R im ersten Versuchsteil genutzt.
Für die Messung der Signalverläufe in den Kabeln wurde ein Osziloskop bentutzt.
Dieses hat einen eingebauten Signalgenerator welcher für die weiteren Versuchsteile genutzt wurde.
