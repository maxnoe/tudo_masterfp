\section{Auswertung}
\label{sec:Auswertung}

\subsection{Messung der Kabeleigenschaften}

\begin{figure}
  \centering
  \includegraphics{lcr.pdf}
  \caption{Widerstand $R$, Kapazität $C$ und Induktivität $L$ für das \texttt{rg085}-Kabel mit einer Länge von \SI{85}{\meter} in Abhängigkeit der Frequenz der angelegten Sinusspannung.}
  \label{fig:lcr}
\end{figure}


\subsection{Messung der Dämpfungskonstante}

Für die Messung der frequenzabhängingen Dämpfung wird eine Rechteckspannung 
jeweils über ein kurzes Kabel (ca.\ \SI{25}{\centi\meter}) und das zu untersuchende \SI{85}{\meter}-Kabel in ein Oszilloskop gespeist.

Da eine Rechteckspannung durch eine Fourierreihe aus vielen Frequenzen erzeugt wird,
kann die Dämpfungskonstante in einer Messung für viele Frequenzen bestimmt werden.
Hierzu werden die gemessenen Spannungen im Oszilloskop Fourier-transformiert.

Die eingespeisten Rechteckspannungen sind in \autoref{fig:attenuation_signal} dargestellt.
Es lässt sich erkennen, dass die Form für das kurze Kabel deutlich \enquote{eckiger} ist als bei dem Signal, dass durch das \SI{85}{\meter}-Kabel geleitet wurde.
Dies deutet auf eine Abschwächung der höheren Frequenzen hin.

\begin{figure}
  \centering
  \includegraphics{attenuation_signal.pdf}
  \caption{%
    Rechteckspannung im Oszilloskop nach Leitung durch das kurze bzw.\ \SI{85}{\meter}-Kabel.%
  }\label{fig:attenuation_signal}
\end{figure}

In \autoref{fig:attenuation_fft} sind die Fourier-transformierten Spannungswerte gegen die Frequenzen aufgetragen.
Die Dämpfungskonstante wird für jedes lokale Maxima im Frequenzraum bestimmt,
da das Oszilloskop bereits die Amplitude $A$ in \si{\deci\bel} ermittelt, ergibt sich die Dämpfungskonstante $α$ zu
\begin{equation}
  α / \si{\deci\bel} = (A_{\SI{85}{\meter}} - A_\text{kurz}) / \si{\deci\bel}
\end{equation}

\begin{figure}
  \centering
  \includegraphics{attenuation_fft.pdf}
  \caption{%
    Fourier-transformiertes Signal mit lokalen Maxima für beide Kabel, sowie die für die lokalen Maxima bestimmte Dämpfungskonstante $α$.
  }\label{fig:attenuation_fft}
\end{figure}

\subsection{Bestimmung der Kabellänge}

\begin{figure}
  \centering
  \includegraphics{length_measurement.pdf}
  \caption{Signal des NIM-Pulses mit Reflektion, der Zeitabstand zwischen den beiden Pulsen wird zur Bestimmung der Kabellänge genutzt.}
  \label{fig:length}
\end{figure}

\subsection{Vermessung von unbekannten Abschlusswiderständen}

\begin{figure}
  \centering
  \includegraphics{unknown.pdf}
  \caption{Signalspannung am Oszilloskop für drei unbekannte Abschlusswiderstände.}
  \label{fig:unknown}
\end{figure}
