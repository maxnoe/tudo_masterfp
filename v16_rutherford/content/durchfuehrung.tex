\section{Versuchsaufbau}

Der Versuchsaufbau besteht im wesentlichen aus der Strahlungsquelle, einer Goldfolie und einem Surface-Barrier Detektor.
Der Aufbau befindet sich in einer Vakuumkammer welche durch eine Drehschieberpumpe evakuiert werden kann.
Die α-Teilchen aus der Quelle werden durch \SI{2}{\mm} Schlitzblenden kollimiert bevor sie auf die Goldfolie treffen.
Der Versuchsaufbau ist in \autoref{fig:aufbau} schematisch dargestellt.

\begin{figure}
  \centering
  \includegraphics[scale=1.2]{aufbau.pdf}
  \caption{%
    Schematische Darstellung des Versuchsaufbaus~\cite{anleitung_v16}, Maße in Millimetern.
  }\label{fig:aufbau}
\end{figure}

\section{Versuchsdurchführung}
Als Quelle für Alpha-Teilchen wird in diesem Versuch \ce{^{241}Am} genutzt.
Dieses Isotop besitzt drei Alpha-Linien\cite{alpha-spectrum}:
\begin{center}
  \begin{tabular}{S[table-format=1.3] @{\,} s S[table-format=1.2]}
    \toprule
    \multicolumn{2}{c}{Energie} & {Anteil} \\
    \midrule
    5.486 & MeV & 0.85 \\
    5.443 & MeV & 0.13 \\
    5.388 & MeV & 0.01 \\
    \bottomrule
  \end{tabular}
\end{center}

Die Probe hatte im Oktober 1994 eine Aktivität von $A_0 = \SI{330}{\kilo\becquerel}$,
hieraus ergibt sich mit der Halbwertszeit von \ce{^{241}Am} von \num{432.5} Jahren eine Aktivität zur Zeit der Versuchsdurchführung \SI{318}{\kilo\becquerel}.
In den folgenden Berechnungen wird nur die Hauptlinie des Spektrums bei \SI{5.486}{MeV} berücksichtigt.

\begin{equation}
  \label{eq:activity}
  A = A_0 \cdot \E^{
    - \ln(2) \frac{t}{T_{\sfrac{1}{2}}}
  } = \SI{318}{\kilo\becquerel}.
\end{equation}

Als Vorbereitungsaufgabe sollte das Bremsvermögen
der Alpha-Teilchen in Luft berechnet bestimmt werden. Daraus solte dann der Kammerdruck bestimmt werden ab dem  ein Energieverlust
bemerkbar wird. In Abbildung~\ref{fig:bethe} ist das Bremsvermögen dargestellt.

\begin{figure}
  \centering
  \includegraphics[width=0.9\textwidth]{./build/plots/bethe_air.pdf}
  \caption{Das Bremsvermögen der Alpha-Teilchen gegen den Kammerdruck. Die Atmosphäre wurde hier als reiner Stickstoff mit einer Ionisationsenergie von \SI{5.408}{\MeV} genähert.}
  \label{fig:bethe}
\end{figure}

Integriert man den Kehrwert des Energieverlusts von der Teilchenenergie $E_{\alpha}$ bis $0$ erhält man die Reichweite der Alphastrahlung.
In \autoref{fig:alpha_range} ist die Reichweite der Alpha-Strahlung für mehrere Drücke aufgetragen.

\begin{figure}
  \centering
  \includegraphics{range_alpha.pdf}
  \caption{Reichweite von Alpha-Strahlung für verschiedene Drücke in einer Stickstoff-Atmosphäre.}
  \label{fig:alpha_range}
\end{figure}

Im vorliegenden Versuchaufbau befinden sich etwa \SI{10}{\cm} Atmosphäre zwischen Quelle und Detektor.
Eine Reichweite von \SI{10}{\centi\meter} wird bei ca.\  \SI{350}{\milli\bar} erreicht.

An den Surface-Barrier Detektor wird ein Osziloskop angeschlossen.
In \autoref{fig:pulse} sind beispielhaft zwei Pulse gezeigt.
Der obere Puls ist ohne Vorverstärker aufgenommen, der untere mit.

\begin{figure}
  \centering
  \includegraphics[width=0.9\textwidth]{./build/plots/pulses.pdf}
  \caption{%
    Vom Oszilloskop aufgenommene Pulse ohne (oben) und mit (unten) Vorverstärkung.
  }%
  \label{fig:pulse}
\end{figure}


\subsection{Bestimmung der Foliendicke}

Zunächst soll die Dicke der Goldfolie durch eine Energieverlustmessung bestimmt werden.

Hierzu wird die mittlere Amplitude der Pulse des
Surface-Barrier-Detektors für verschiedene Drücke bestimmt.
Die Messung wird einmal mit und einmal ohne Folie durchgeführt.

Da beim Surface-Barrier-Detektor die Höhe der Pulse proportional zur Energie der eintreffenden Teilchen ist,
kann über den Unterschied zwischen der Messung mit und ohne Folie den Energieverlust $\increment E$ ermitteln.

In \autoref{fig:thickness} sind die beiden Messreihen aufgetragen.
Über eine lineare Extrapolation wird die Energie der α-Teilchen bei einem Druck von $0$ bestimmt, also für keinerlei Energieverlust in der Atmosphäre.

Für die Parameter der linearen Regression ergibt sich:
\input{fit_results.tex}
Der Energieverlust ergibt sich zu:
\begin{equation}
  \increment E = E_α \cdot \left(1 - \frac{b_\text{ohne}}{b_\text{mit}}\right)
  = \input{delta_E.tex}
\end{equation}

Die Bethe–Bloch-Formel \eqref{eq:bethe} wird zu
\begin{equation}
  \frac{\increment E}{\increment x} = \frac{4 \pi^4 z^2 n e^4}{m_0 v^2 (4 \pi \epsilon_0)^2} \ln \frac{2 m_0 v^2}{I}
\end{equation}
genähert.
Für die Geschwindigkeit wird die mittlere Geschwindigkeit im Gold eingesetzt, die sich über
\begin{equation}
  \bar{v} = \sqrt{\frac{2 (E + \sfrac{\increment E}{2})}{m_α}}\label{eq:shit-approximation}
\end{equation}
ergibt.
Die Dicke der Goldfolie ergibt sich nun zu
\begin{equation}
  \increment x_{\ce{Au}} = \input{thickness.tex} .
\end{equation}

\begin{figure}
  \centering
  \includegraphics{gold_thickness.pdf}
  \caption{%
    Mittlere Pulshöhe am Oszilloskop gegen Kammerdruck für die Messungen mit bzw.\ ohne Folie.
    Die durchgezogenen Linien zeigen das Ergebnis der linearen Regression.%
  }\label{fig:thickness}
\end{figure}

\subsection{Bestimmung des Streuquerschnittes}
\label{subs:cross_section}

Zur Bestimmung des Streuquerschnittes werden Zählraten bei verschiedenen Winkeln $\theta$ bei evakuierter Kammer gemessen.
Die gemessenen Werte sind in \autoref{tab:counts} aufgelistet. Zur Umrechnung in den differentiellen Streuquerschnitt wird zunächst
aus der in \eqref{eq:activity} berechneten Quellaktivität von \SI{318}{\kilo \becquerel} abgeschätzt wieviele Teilchen von der Quelle
am Detektor ankommen. Dazu wurde angenommen, dass die Quelle homogen in alle Richtungen abstrahlt.
Aus der aktiven Detektoberfläche hinter dem Spalt $A_{Det} = \SI{2}{\mm} \cdot \SI{10}{\mm} = \SI{20}{\milli \meter \square}$ ergibt sich
\begin{equation}
  \label{eq:rate}
  N_\text{Quelle} = \frac{A_{Det}}{4  \pi  R^2} \cdot \SI{318}{\kilo \becquerel} = \SI{389.6}{\becquerel}
\end{equation}
mit dem Abstand von Detektor zur Quelle $R = \SI{10,1}{\centi \meter}$.
Der Raumwinkel $\Delta \Omega$ den der Spalt von der Streuposition in der Goldfolie einnimmt ergibt sich aus dem Abstand der Folie zum Detektor
$l = \SI{45}{\mm}$ und der Spaltgröße $dx = \SI{2}{\mm}$ $dy = \SI{10}{\mm}$ zu
\begin{equation}
\increment \Omega = \arctan(\frac{dy}{2  l})  \arctan(\frac{dx}{2  l}) \cdot 4.
\end{equation}

Die Anzahl der Streupartner in der Goldfolie kann abgeschätzt werden durch die Dichte von Gold $\rho_{\ce{Au}} =  \SI{19.3}{\gram \per \cubic \centi \meter}$
die Dicke der Folie $d_{\ce{Au}}  = \SI{2}{\micro \meter}$, die Avogadro-Konstante $N_\text{A} = \SI{6,022E23}{\per \mol}$ und die Molare Masse
von Gold $M_{\ce{Au}} = \SI{197}{\gram \per \mol}$ zu
\begin{equation}
  n_0 = \frac{\rho_{\ce{Au}}  d_{\ce{Au}} N_\text{A} }{M_{\ce{Au}}}.
\end{equation}

Damit lässt sich der differentielle Streuquerschnitt berechnen zu
\begin{equation}
  \dd{\sigma}{\Omega} = \frac{N_\text{Messung}}{n_0 N_\text{Quelle} \increment \Omega}.
\end{equation}

Die Ergebnisse sind zusammen mit der Theoriekurve in \autoref{fig:cross_section} dargestellt.
\begin{figure}
  \centering
  \includegraphics{cross_section.pdf}
  \caption{%
    Theoretischer und gemessener Streuquerschnitt von Alpha-Teilchen an einer Goldfolie.
  }\label{fig:cross_section}
\end{figure}
