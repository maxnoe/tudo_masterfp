\section{Diskussion}\label{sec:Diskussion}


Ein Literaturwert für die Aktivierungsenergie beträgt:
\begin{equation}
  W_\text{lit} = \SI{0.66}{\electronvolt} \quad \text{\cite{dipoles}}
\end{equation}

Dies scheint verträglich mit unseren Messwerten.

Die Relaxationszeit weicht jedoch deutlich vom Literaturwert ab.
\begin{equation}
  \tau_\text{lit} = \SI{4E-10}{\second} \quad \text{\cite{dipoles}}
\end{equation}

Eine große Fehlerquelle bei diesem Versuch wird die Messung des Stromes durch das
Ampermeter sein. Die Anzeige schien sich während der Messung instabil zu verhalten.
Auch ist die Wahl von Punkten die in den Fit eingehen und die Wahl von $T'$ eher beliebig
wirkt sich aber stark auf das Endergebniss aus.  
